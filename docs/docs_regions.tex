\chapter{HiCEnterprise regions}
\label{regions}

Regions part of package can be used to identify long-range interacting regions and creating interaction profiles based
on Hi-C data. It is based on the method described by (Won, 2016).

The significance of the individual pairwise interaction between bins is calculated as the probability of observing a
stronger contact under the fitted Weibull distribution matched by chromosome and distance. FDR (False Discovery Rate) is
used for correcting for multiple testing.

\section{Input files}
\label{sec:input}

To run the point interactions analysis you need:
\begin{itemize}
    \item at least one Hi-C map in numpy format for an N chromosome (named mtx-N-N.npy)
    \item BED file with coordinates of regions to extract interaction profiles for (or Enhancer Atlas FASTA file - it has positions of the enhancer region in fasta header of each sequence).
\end{itemize}

All coordinates should be provided in bp. Analysis results will be more reliable if you provide multiple Hi-C maps that
are biological replicates.

\subsubsection{Example files}
Example files are provided in \lstinline{HiCEnterprise/HiCEnterprise/tests/test_files} directory in \lstinline{maps} and \lstinline{enhseq}.

\section{Usage}

\subsection{Basics}
Overall, the usage looks like this:
\begin{lstlisting} 
    HiCEnterprise regions regions_options
\end{lstlisting}

To see the available options:
\begin{lstlisting}
HiCEnterprise regions -h
\end{lstlisting}

\subsubsection{Example runs}
There are a few required arguments for the analysis.
\begin{itemize}
    \item Path to folders with Hi-C maps in a numpy format: \\
\lstinline{--hic_folders ./map40kb_replicate1 ./map40kb_replicate2 \ldots}
    \item File with coordinates of regions to extract interaction profiles for: \\
\lstinline{--region_file regions.bed}
    \item Name of the chromosome to run the analysis for:  \lstinline{--chr 22}
    \item Resolution (bin size) of Hi-C maps in bp (all maps provided should be of the same resolution): \lstinline{--bin_res 40000}
\end{itemize}

For information on formats of files see~\ref{sec:input}.



Run with only required arguments:
\begin{lstlisting}
HiCEnterprise regions --hic_folders ./map40kb_replicate1 
./map40kb_replicate2 --region_file ./regions.bed --chr 22
--bin_res 40000
\end{lstlisting}

\subsubsection{Visualisation}

If you want to plot the results, add --plotting argument to the command. It is possible to choose if you want to plot
with matplotlib (mpl) or rpy2 and R (rpy2). Resulting figures can be found in directory provided by --figures\_folder
option.

Example plotting with matplotlib:
\begin{lstlisting}
HiCEnterprise regions --hic_folders ./map40kb_replicate1 
./map40kb_replicate2 --region_file ./regions.bed --chr 22 --bin_res
40000 --plotting mpl --figures_folder ./my_folder_for_figures
\end{lstlisting}

If you don not provide the --figures\_folder argument, ./figures directory will be created and the plots will be put
there.

\paragraph{Example usage files}

There are also some files with example usage files provided in HiCEnterprise/example\_usage directory i.e.:
\begin{itemize}
    \item TSS enrichment analysis for found predictions of interactions
    \item generating random bins
    \item extracting unique bins and adjusting statistics for comparison between two analyses
\end{itemize}

Those additional scripts may generate new directories and results. For now, as they are not an official part of the
package they are not discussed in this documentation.

\section{Output files}

\subsection{stats folder}
This folder contains the results from the analysis. Result filenames consist of regions file name, map folders names,
either 'stats' or 'significant', chromosome name, number of bins considered, resolution and statistical cutoff
threshold. Example: regions-map40kb\_replicate1-map40kb\_replicate2-stats22-200x40000bp-0\_01.txt.

\subsubsection{'stats' files}
Stats files are text files that contain statistics derived from analysis. They provide the number of regions (bin
regions) in the analysis, how many had predictions and some simple statistics.

\subsubsection{'significant' files}
Files with 'significant' in name contain the statistically significant (with given FDR threshold) interactions derived
from the analysis. They can be created in three formats (see~\ref{sec:arguments}):
\paragraph{custom txt format}
Txt contains bin and regions coordinates and corresponding predictions of interactions with q-values.
\paragraph{modified BED format}
3 first columns are coordinates of the predicted interaction, 4th column encodes coordinates of region that interacts,
5th column is score (-log10 q-value).
\paragraph{modified GFF format}
Modified GFF contains columns as follows:
\begin{itemize}
    \item chromosome name
    \item program name
    \item type: group (group of regions and their predictions), region, prediction
    \item start coordinate
    \item end coordinate
    \item score (-log10 q-value) for predictions
    \item strand (none, '.')
    \item frame (none, '.')
    \item ID of Parent (group of regions and their predictions) with chromosome name and bin from original assembly
\end{itemize}

Results in GFF files can be remapped from the original assembly to another one with argument --remap (see ~\ref{sec:arguments}).
GFF files are suitable for a representation in Jbrowse with modified CSS (thanks to Karolina Sienkiewicz).

\subsection{figures folder}
In this folder you will find the plotted figures if the --plotting argument was used. Interaction profiles are
extracted to PDF files. Files are named by the regions file name and additional plotting arguments (--num\_regs,
--section). Each interaction profile is composited of weighted log Intensity, -log10 p-values and -log10 q-values.

\subsection{pickles folder}
Pickles folder contains pickle files (generated by pickle python library) with saved parts of results that may be useful
for running new analyses on the same data (i.e.~same Hi-C maps but other regions). Weibull fitting takes a while, so
saving the parameters of the distribution for a given map, chromosome and distance can save a lot of time in future
calculations.

\section{All arguments listed alphabetically}
\label{sec:arguments}
\begin{itemize}
\item \href{https://www.python.org}{Python} version 2.7 or 3.5/6
\item required python packages are provided in requirements.txt file.
\item if you want to plot through R, you need have R installed with needed packages.
\item -a, --section : Section of bp for the plot
\item -b, --bin\_res : REQUIRED Resolution (size of the bins on the Hi-C maps) in bp i.e.~10000 for 10kb resolution. All maps
should be of same resolution.
\item -c, --chr : REQUIRED Name of the chromosome for which to extract domain interactions.
\item -f, --figures\_folder : Folder to save the figures (plots) in. Default is '../figures/'.
\item -m, --hic\_folders : REQUIRED Folder/folders in which the Hi-C data is stored with file names in format mtx-N-N.npy, where N =
chromosome name
\item -n, --num\_bins : Number of bins left and right to be considered when extracting the interaction profile.
\item --num\_regs : Number of regions to plot
\item -p, --pickled\_folder : Folder with pickled files (to load from/save in). Default is '../pickles/'.
\item --plotting : If there should be plotting, and if so should it be with rpy2 or matplotlib. Options: mpl, rpy2. Default is mpl.
\item -r, --region\_file : REQUIRED Any tab-delimited BED file or FASTA File from EnhancerAtlas with regions and their positions.
\item --regs\_algorithm : Algorithm for sorting regions to bins. Options: all, one. Default is all. 'all' assigns the region to all bins that if fits into; 'one' chooses the bin in which most of the region is.
\item --remap : Assembly that maps and regions are currently in, and assembly in which to save the stats in. Format:
assembly:new\_assembly i.e.~hg\_19:hg\_38. Currently only available for GFF files.
\item -s, --stats\_folder : Folder to save the statistics and significant points in. Default is '../stats/'.
\item --single\_sig : If the single map significant points should be saved or not.
\item --stat\_formats : In which file formats to save statistics. It is possible to provide multiple formats. Available:
txt, bed, gff. Default is only txt.
\item -t, --threshold : Statistical cutoff threshold. Default is 0.01.
\end{itemize}



