\chapter{HiCEnterprise domains}
\label{domains}

In the domain part of the package the significance of interaction between domains A and B is calculated as the
probability of observing a stronger contact under the fitted distribution. There are three distributions available:
Hypergeometric (default), Negative Binomial and Poisson.

\section{Available distributions}

\subsection{Poisson}
Another approach is to assume that the contacts are distributed according to the Poisson distribution with the $\lambda$ parameter depending on the distance from the diagonal. In this case, for every domain pair, we estimate the average contact frequency $\lambda_d$ in the diagonals corresponding to the distance between them and score the given domain contact against the Poisson distribution with the mean and variance $\lambda_d$. Poisson distribution is used as default by the program.

\subsection{Hypergeometric}

\begin{equation}
Pvalue = 1 - \sum_{i=0}^{k-1} \frac{{a\choose i}{N-a\choose b-i }}{{N\choose b}}
\end{equation}

Here $N$ (population size) is the sum of all frequencies in the Hi-C map, $a$ is the sum from the column of domain A and
$b$ is the sum from the row of domain B (number of success states in the population and number of draws). $k$ is the number of contacts between domains A and B. This idea
and original code was developed by Irina Tuszyńska and presented in (Niskanen et al. n.d.).

\subsection{Negative Binomial}
The negative binomial distribution is slightly more general than the Poisson distribution discussed in the previous section. In particular it allows for overdispersion of the data. In this case, for each pair of domains at the distance $d$ we estimate the mean contact frequency $\mu_d$ and variance $\sigma^2_d \geq \mu$ from all values in the respective diagonals and then score the contact significance of this domain pair against the negative binomial distribution with the estimated mean and variance. 

\section{Input files}
\label{sec:input2}

To run the domain-domain interactions analysis you need:
\begin{itemize}
    \item one Hi-C map in numpy format for one chromosome
    \item file with domains definition in the format: domain\_id chromosome\_name start\_coordinate end\_coordinate (space
    delimited). Optionally fifth column may contain \lstinline{sherpa}\_level.
\end{itemize}

All coordinates should be provided in bp.

\subsubsection{Example files}
Example files are provided in \lstinline{HiCEnterprise/HiCEnterprise/tests/test_files} directory in \lstinline{maps} and \lstinline{doms}. 

Overall, the usage looks like this:

\begin{lstlisting}
HiCEnterprise domains domains_options
\end{lstlisting}


To see the available arguments:
\begin{lstlisting}
HiCEnterprise domains -h
\end{lstlisting}


\subsubsection{Example runs}
There are a few required arguments for the analysis.
\begin{itemize}
    \item Path to the Hi-C map in a numpy format: \lstinline{--hic_map ./map40kb.npy}
    \item File with domain definition: \lstinline{--domain_file domains.txt}
    \item Name of the chromosome to run the analysis for: \lstinline{--chr 22}
    \item Resolution (bin size) of Hi-C maps in bp (all maps provided should be of the same resolution): \lstinline{--bin_res 40000}
\end{itemize}

For information on formats of files see~\ref{sec:input2}.

Run with only required arguments:
\begin{lstlisting}
HiCEnterprise domains --hic_map ./map40kb.npy --domain_file
./domains.txt --chr 22 --bin_res 40000
\end{lstlisting}


\subsubsection{Visualisation} 

If you want to plot the results with matplotlib, add --plotting argument to the command. Resulting figures can be found
in director provided by --figures\_folder option.

In example:
\begin{lstlisting}
HiCEnterprise domains --hic_map ./map40kb.npy --domain_file
./domains.txt --chr 22 --bin_res 40000 --plotting
--figures_folder ./my_folder_for_figures
\end{lstlisting}


If you don not provide the --figures\_folder argument, ./figures directory will be created and the plots will be put
there. There are additional options that you can use to change your visualization, i.e. if you want to change colors that
represent the Hi-C map and domain-domain interactions, you can use --hic\_color and --interactions\_color with any
available palette from https://matplotlib.org/api/pyplot\_summary.html.

\section{Output files}

\subsection{stats folder}
This folder contains the results from the analysis. Result filenames consist of domains file name, map name,
'stats' or 'corr\_stats', chromosome name, statistical cutoff threshold, and distribution name. Example:
domains-map40kb-stats22-0\_01-hypergeom.txt.

\subsection{figures folder}
In this folder you will find the plotted figures if the --plotting argument was used. Interaction maps are
extracted to PNG files. Files are named by the map name, "corr" if the FDR corrected data is on the plot, domains file
name, and distribution name. Example: map40kb-corr\_domains-hypergeom.png

More interesting examples with real HiC maps are available on the HiCEnterprise site http://regulomics.mimuw.edu.pl/wp/hicenterprise/

\section{All arguments listed alphabetically}
\label{sec:arguments2}
\begin{itemize}
   \item --all\_domains : Stop remove pericentromeric domains and domains with rare interdomains contacts, where a mean
    number of contacts in one row is less than n*numer\_of\_domains. n = 1 and can be changed by --interact\_indomain parameter
    \item -b, --bin\_res : REQUIRED Resolution (size of the bins on the Hi-C maps) in bp i.e.~10000 for 10kb resolution.
    \item -c, --chr : REQUIRED Name of the chromosome for which to extract domain interactions.
    \item --distribution : The distribution on which you would like to base the identification of domain-domain
    interactions. Available: hypergeom, negbinom, poisson. Default is hypergeom.
    \item -d, --domain\_file : REQUIRED Text file with domain definition in the format: \\
    dom\_id(int) chromosome(int) dom\_start(bp) dom\_end(bp) sherpa-lvl(optional).
    \item -e, --ticks\_separation : Frequency of ticks on the plot.
    \item -f, --figures\_folder : Folder to save the figures (plots) in. Default is './figures/'.
     \item  -g --interact\_indomain : Multiplier of domains\_number, that is a threeshold for neglecting pericentromeric domains. 
    Mutualy exclusive with --all\_domains option. Default = 1, higher number - more domains will be removed.
    \item -l, --plot\_title : The title of the plot. If it contains spaces, use quotation marks. Default is
    "Interactions".
    \item -m, --hic\_map : Path to the single-chromosome Hi-C map in numpy format.
    \item -n, --hic\_name : Name to use for Hi-C map. Default is the name of the file.
    \item -o, --hic\_color : The color of HiC map. You can choose the palette from Colormaps available here:
    https://matplotlib.org/api/pyplot\_summary.html. Recommended are Reds, Blues, YlOrBr, PuBu. Default is Greens.
    \item --plotting : If results should be plotted. matplotlib library is required.
    \item -r, --interactions\_color : The color of interactions. You can choose the palette from Colormaps available
    here: https://matplotlib.org/api/pyplot\_summary.html. Recommended are Reds, Blues, YlOrBr, PuBu. Default is YlOrBr.
    \item -s, --stats\_folder : Folder to save the statistics and significant points in. Default is './stats/'.
    \item --sherpa\_lvl : If there are sherpa levels in the file and which one to use.
    \item -t, --threshold :  Statistical cutoff threshold. Default is 0.01.
\end{itemize}
