\chapter{HiCEnterprise tads}
\label{domains}

In the domain part of the package the significance of interaction between domains A and B is calculated as the
probability of observing a stronger contact under the fitted Hypergeometric distribution, where N (population size) is
the sum of all frequencies in the Hi-C map, a is the sum from the column of domain A and b is the sum from the row of
domain B (number of success states in the population and number of draws). This idea and original code was developed
by Irina Tuszynska and presented in (Niskanen et al. n.d.).

\section{Input files}
\label{sec:input2}

To run the point interactions analysis you need:
\begin{itemize}
    \item one Hi-C map in numpy format for an N chromosome (named mtx-N-N.npy)
    \item file with domains definition in the format: domain id, chromosome name, start coordinate, end coordinate. Optionally fifth column may contain \lstinline{sherpa} level.
\end{itemize}

All coordinates should be provided in bp.

\subsubsection{Example files}
Example files are provided in \lstinline{HiCEnterprise/HiCEnterprise/tests/test_files} directory in \lstinline{maps} and \lstinline{doms}.

Overall, the usage looks like this:

\begin{lstlisting}
HiCEnterprise domains domains_options
\end{lstlisting}


To see the available options:
\begin{lstlisting}
HiCEnterprise domains -h
\end{lstlisting}


\subsubsection{Example runs}
There are a few required arguments for the analysis.
\begin{itemize}
    \item Path to folder with Hi-C map in a numpy format: \lstinline{--hic_folder ./map40kb}
    \item File with domain definition: \lstinline{--domain_file domains.txt}
    \item Name of the chromosome to run the analysis for: \lstinline{--chr 22}
    \item Resolution (bin size) of Hi-C maps in bp (all maps provided should be of the same resolution): \lstinline{--bin_res 40000}
\end{itemize}

For information on formats of files see~\ref{sec:input2}.

Run with only required arguments:

\begin{lstlisting}
HiCEnterprise domains --hic_folder ./map40kb --domain_file
./domains.txt --chr 22 --bin_res 40000
\end{lstlisting}


\subsubsection{Visualisation} % TODO !!!!

If you want to plot the results with matplotlib, add --plotting argument to the command. Resulting figures can be found in directory
provided by --figures\_folder option.

In example:
\begin{lstlisting}
HiCEnterprise domains --hic_folder ./map40kb --domain_file
./domains.txt --chr 22 --bin_res 40000 --plotting
--figures_folder ./my_folder_for_figures
\end{lstlisting}


If you don not provide the --figures\_folder argument, ./figures directory will be created and the plots will be put
there.

\section{Output files}

\subsection{stats folder}
This folder contains the results from the analysis. Result filenames consist of domains file name, map folder names,
'stats', chromosome name and statistical cutoff. % TODO what about resolution
threshold. Example: domains-map40kb-stats22-0\_01.txt.

\subsection{figures folder}
In this folder you will find the plotted figures if the --plotting argument was used. Interaction maps are
extracted to PNG files. Files are named by the regions file name, maps name and other arguments.

\section{All arguments listed alphabetically}
\label{sec:arguments2}
\begin{itemize}
    \item -b, --bin\_res : REQUIRED Resolution (size of the bins on the Hi-C maps) in bp i.e.~10000 for 10kb resolution.
All maps should be of same resolution.
\item -c, --chr : REQUIRED Name of the chromosome for which to extract domain interactions.
    \item -d, --domain\_file : REQUIRED Text file with domain definition in the format: \\
    dom\_id(int) chromosome(int) dom\_start(bp) dom\_end(bp) sherpa-lvl(optional).
    \item -f, --figures\_folder : Folder to save the figures (plots) in. Default is '../figures/'.
    \item -m, --hic\_folder : REQUIRED Folder in which the Hi-C data is stored with file names in format mtx-N-N.npy, where
N = chromosome name.
\item --plotting : If results should be plotted. matplotlib library is required.
    \item -s, --stats\_folder : Folder to save the statistics and significant points in. Default is '../stats/'.
    \item --sherpa\_lvl : If there are sherpa levels in the file and which one to use.
\item -t, --threshold :  Statistical cutoff threshold. Default is 0.01.
\end{itemize}




